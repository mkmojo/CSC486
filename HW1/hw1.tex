%%%%%%%%%%%%%%%%%%%%%%%%%%%%%%%%%%%%%%%%%
% Structured General Purpose Assignment
% LaTeX Template
%
% This template has been downloaded from:
% http://www.latextemplates.com
%
% Original author:
% Ted Pavlic (http://www.tedpavlic.com)
%
% Note:
% The \lipsum[#] commands throughout this template generate dummy text
% to fill the template out. These commands should all be removed when 
% writing assignment content.
%
%%%%%%%%%%%%%%%%%%%%%%%%%%%%%%%%%%%%%%%%%

%-------------------------------------------------------------------------------------
%	PACKAGES AND OTHER DOCUMENT CONFIGURATIONS
%-------------------------------------------------------------------------------------

\documentclass{article}

\usepackage{fancyhdr} % Required for custom headers
\usepackage{lastpage} % Required to determine the last page for the footer
\usepackage{extramarks} % Required for headers and footers
\usepackage{graphicx} % Required to insert images
\usepackage{mathtools} % Required to use \text 

% Margins
\topmargin=-0.45in
\evensidemargin=0in
\oddsidemargin=0in
\textwidth=6.5in
\textheight=9.0in
\headsep=0.25in 

\linespread{1.1} % Line spacing

% Set up the header and footer
\pagestyle{fancy}
\lhead{\hmwkAuthorName} % Top left header
\chead{\hmwkClass\ (\hmwkClassInstructor\ \hmwkClassTime): \hmwkTitle} % Top center header
\rhead{\firstxmark} % Top right header
\lfoot{\lastxmark} % Bottom left footer
\cfoot{} % Bottom center footer
\rfoot{Page\ \thepage\ of\ \pageref{LastPage}} % Bottom right footer
\renewcommand\headrulewidth{0.4pt} % Size of the header rule
\renewcommand\footrulewidth{0.4pt} % Size of the footer rule

\setlength\parindent{0pt} % Removes all indentation from paragraphs

%-------------------------------------------------------------------------------------
%	DOCUMENT STRUCTURE COMMANDS
%	Skip this unless you know what you're doing
%-------------------------------------------------------------------------------------

% Header and footer for when a page split occurs within a problem environment
\newcommand{\enterProblemHeader}[1]{
\nobreak\extramarks{#1}{#1 continued on next page\ldots}\nobreak
\nobreak\extramarks{#1 (continued)}{#1 continued on next page\ldots}\nobreak
}

% Header and footer for when a page split occurs between problem environments
\newcommand{\exitProblemHeader}[1]{
\nobreak\extramarks{#1 (continued)}{#1 continued on next page\ldots}\nobreak
\nobreak\extramarks{#1}{}\nobreak
}

\setcounter{secnumdepth}{0} % Removes default section numbers
\newcounter{homeworkProblemCounter} % Creates a counter to keep track of the number of problems

\newcommand{\homeworkProblemName}{}
\newenvironment{homeworkProblem}[1][Problem \arabic{homeworkProblemCounter}]{ % Makes a new environment called homeworkProblem which takes 1 argument (custom name) but the default is "Problem #"
\stepcounter{homeworkProblemCounter} % Increase counter for number of problems
\renewcommand{\homeworkProblemName}{#1} % Assign \homeworkProblemName the name of the problem
\section{\homeworkProblemName} % Make a section in the document with the custom problem count
\enterProblemHeader{\homeworkProblemName} % Header and footer within the environment
}{
\exitProblemHeader{\homeworkProblemName} % Header and footer after the environment
}

\newcommand{\problemAnswer}[1]{ % Defines the problem answer command with the content as the only argument
\noindent\framebox[\columnwidth][c]{\begin{minipage}{0.98\columnwidth}#1\end{minipage}} % Makes the box around the problem answer and puts the content inside
}

\newcommand{\homeworkSectionName}{}
\newenvironment{homeworkSection}[1]{ % New environment for sections within homework problems, takes 1 argument - the name of the section
\renewcommand{\homeworkSectionName}{#1} % Assign \homeworkSectionName to the name of the section from the environment argument
\subsection{\homeworkSectionName} % Make a subsection with the custom name of the subsection
\enterProblemHeader{\homeworkProblemName\ [\homeworkSectionName]} % Header and footer within the environment
}{
\enterProblemHeader{\homeworkProblemName} % Header and footer after the environment
}
   
%-------------------------------------------------------------------------------------
%	NAME AND CLASS SECTION
%-------------------------------------------------------------------------------------

\newcommand{\hmwkTitle}{Assignment\ \#1} % Assignment title
\newcommand{\hmwkDueDate}{Tue,\ Feb\ 25,\ 2014} % Due date
\newcommand{\hmwkClass}{CSC\ 486} % Course/class
\newcommand{\hmwkClassTime}{12:25am} % Class/lecture time
\newcommand{\hmwkClassInstructor}{Muthuv} % Teacher/lecturer
\newcommand{\hmwkAuthorName}{Qiyuan Qiu} % Your name

%-------------------------------------------------------------------------------------
%	TITLE PAGE
%-------------------------------------------------------------------------------------

\title{
\vspace{2in}
\textmd{\textbf{\hmwkClass:\ \hmwkTitle}}\\
\normalsize\vspace{0.1in}\small{Due\ on\ \hmwkDueDate}\\
\vspace{0.1in}\large{\textit{\hmwkClassInstructor\ \hmwkClassTime}}
\vspace{3in}
}

\author{\textbf{\hmwkAuthorName}}
\date{} % Insert date here if you want it to appear below your name

%-------------------------------------------------------------------------------------

\begin{document}

\maketitle

%-------------------------------------------------------------------------------------
%	TABLE OF CONTENTS
%-------------------------------------------------------------------------------------

%\setcounter{tocdepth}{1} % Uncomment this line if you don't want subsections listed in the ToC

\newpage
\tableofcontents
\newpage

%-------------------------------------------------------------------------------------
%	PROBLEM 1
%-------------------------------------------------------------------------------------

% To have just one problem per page, simply put a \clearpage after each problem

\begin{homeworkProblem}[Problem 1]

\problemAnswer{ % Answer
%\begin{center}
%\includegraphics[width=0.75\columnwidth]{example_figure} % Example image
%\end{center}
Assume the three color we can use are red, green, blue. 
Randomly assign each node one of the three colors. That is the probability model we have is 
\[ VertexColor =  
    \left\{ 
        \begin{array}{l l}
            Red \quad 1/3 \\
            Green \quad 1/3\\
            Blue \quad 1/3
        \end{array} 
    \right.
\]

For the graph $ G = (V, E) $, we number all the edges from $1$ to $|E|$. Let $R.V$ denote the satisfiability of edge $i$. As a result we have 

\[ Z_i =
    \left\{ 
        \begin{array}{l l}
            1 \quad \text{if edge i is satisfied}\\
            0 \quad \text{if edge i is not satisfied}
        \end{array} 
    \right.
\]
$$
    Pr[Z_i = 1] = 1 - 3 * (1/3 * 1/3) = 2/3
$$
The expectation being, 
\begin{equation*}
    \label{eq:E_Z_i}
    \begin{split}
        E[Z_i] &= 0 * Pr[Z_i = 0] + 1 * Pr[Z_i = 1] \\ 
        & = Pr[Z_i = 1] \\
        & = 2/3
    \end{split}
\end{equation*}

Let $Z$ denote the total number of edges in $G$ that are satisfied. 
\begin{equation*}
    \begin{split}
    E[Z]&= E[Z_1 + Z_2 + \hdots + Z_{|E|}]  \\
        &= \sum\limits_i^{|E|} E[Z_i] \\
        &= 2/3 |E|\ge 2/3 c^*
    \end{split}
\end{equation*}
}
\end{homeworkProblem}

%-------------------------------------------------------------------------------------
%	PROBLEM 2
%-------------------------------------------------------------------------------------

\begin{homeworkProblem}[Problem 2] % Custom section title

%--------------------------------------------


\problemAnswer{ % Answer
    Let $R.V$ $X_i$ denote the person intends to vote for D. $Y_i$ denotes the person intends to vote for R.
\[ X_i =
    \left\{ 
        \begin{array}{l l}
            1 \quad 99/100\\
            0 \quad 1/100
        \end{array} 
    \right.
\]
\[ Y_i =
    \left\{ 
        \begin{array}{l l}
            1 \quad 1/100\\
            0 \quad 99/100
        \end{array} 
    \right.
\] 
    Let $Z$ denote the total number of people actually vote for $D$.
$$
Z = X_1 + X_2 + \hdots + X_{80,000} + Y_1 + Y_2 + \hdots + Y_{20,000}
$$
\begin{equation*}
\begin{split}
    E[Z] &= \sum\limits_{i=1}^{80,000}E[X_i] + \sum\limits_{i=1}^{20,000}E[Y_i] \\
         &= 80000 * 99/100 + 20000 * 1/100 \\ 
         &= 79400
\end{split}
\end{equation*}
}

%--------------------------------------------

\end{homeworkProblem}

%-------------------------------------------------------------------------------------
%	PROBLEM 3
%-------------------------------------------------------------------------------------

\begin{homeworkProblem}[Problem 3 ] % Roman numerals

%--------------------------------------------

\begin{homeworkSection}{\homeworkProblemName:~(a)} % Using the problem name elsewhere
\problemAnswer{ % Answer
For the purpose of contradiction, assume $P_i, P_j \in S$ and $P_i, P_j$ conflict with each other, i.e. $\exists E(P_i, P_j)$. \\ 

Based on the protocol if $P_i \in S$, we can infer that $P_j \not \in S$. Which implies that $\not \exists E(P_i, P_j)$. \\ 

Therefore our assumption is wrong, so the $S$ is conflict free.

Let $Z$ denotes the number of processes in set $S$. 
$$
Z = X_1 + X_2 + \hdots + X_n
$$
Before we calculate $E[Z]$, lets find out $E[X_i]$.
\begin{equation*}
    \begin{split}
        E[X_i] &= 0 * Pr[X_i = 0] + 1 * Pr[X_i = 1] \\
          &= Pr[X_i = 1]
    \end{split}
\end{equation*}

Because for $X_i$ to be finally in the set $S$, all the $d$ processes conflicts with $X_i$ should not have choose to be in set $S$, therefore.
\begin{equation*}
    \begin{split}
        Pr[X_i = 1] &= 1/2 * (1/2)^d \\ 
                    &=(1/2)^{d+1}
    \end{split}
\end{equation*}

Hence, 
\begin{equation*}
    \begin{split}
        E[Z] &= E[X_1 + X_2 + \hdots + X_n]\\
             &= \sum\limits_{i=1}^n E[X_i] \\
             &= n(1/2)^{d+1}
    \end{split}
\end{equation*}

}

\end{homeworkSection}

%--------------------------------------------

\begin{homeworkSection}{\homeworkProblemName:~(b)}

\problemAnswer{ % Answer
From section (a) we know that 
\begin{equation*}
    \begin{split}
        Pr[X_i = 1] &= p * (1 - p)^d  
    \end{split}
\end{equation*}
Expectation for $X_i$ becomes:
\begin{equation*}
    \begin{split}
        E[X_i] &= 0 * Pr[X_i = 0] + 1 * Pr[X_i = 1] \\
          &= Pr[X_i = 1] \\ 
          &= p * (1 - p)^d 
    \end{split}
\end{equation*}
Following this, the expected size of set $S$ would be:
\begin{equation*}
    \begin{split}
        E[Z] &= E[X_1 + X_2 + \hdots + X_n]\\
             &= \sum\limits_{i=1}^n E[X_i] \\
             &= n * p * (1-p)^d
    \end{split}
\end{equation*}

Let $f(p)$ denote the expected size, set the derivative to $0$ we have:
\begin{equation*}
    \begin{split}
        \frac{d}{dp} f = -n(1-p)^{-1+d}(-1+p+dp) &= 0 
    \end{split}
\end{equation*}
$f^*$ is achieved at $p = 1/(1+d)$, so the expected size of set $S$ is
$$
E[|S|] = E[z] = f^* = n(\frac{1}{1+d})(1 - \frac{1}{1+d})^d
$$

}
\end{homeworkSection}

%--------------------------------------------

\end{homeworkProblem}

%-------------------------------------------------------------------------------------
%	PROBLEM 4
%-------------------------------------------------------------------------------------

\begin{homeworkProblem}[Problem 7] % Roman numerals
    \begin{homeworkSection}{\homeworkProblemName:~(a)}
    \problemAnswer{ % Answer
        Let $Z$ denote the number of clauses that are satisfied.
        \[
            Z = \sum\limits_{i=1}^{k} C_i
        \]
        \[
            E[Z] = \sum\limits_{i=1}^{k} E[C_i]
        \]
        We know that:
        \begin{equation*}
            \begin{split}
                E[C_i] &= 1 * Pr[C_i = 1] * 0 * Pr[C_i = 0] \\
                       &= Pr[C_i = 1] 
            \end{split}
        \end{equation*}
        Let $t_{c_i}$ be the number of variables in clause $C_i$

        \[
            Pr[C_i = 1] = Pr[C_i \  \text{is satisfied}] = 1 - (1/2)^{t_{c_i}}
        \]
        Because we know that $1 \le t_{c_i} \le n$, so we have:
        \begin{equation}
        \label{eq:1}
            \begin{split}
                E[(1/2)^{t_{c_i}}] \le E[(1/2)^1] = (1/2)^1
            \end{split}
        \end{equation}
        The expectation of $Z$ is:
        \begin{equation*}
            \begin{split}
                E[Z] &= \sum\limits_{i = 1}^k E[1 - (1/2)^{t_{c_i}}] \\ 
                     &= k - \sum\limits_{i = 1}^k E[(1/2)^{t_{c_i}}] \\
                     &\ge k - (k * (1/2)^1) \text{\quad Because of  $\eqref{eq:1}$}\\
                     &= (k/2)
            \end{split}
        \end{equation*}
    }
    \end{homeworkSection}
\end{homeworkProblem}

%-------------------------------------------------------------------------------------

\end{document}
